\documentclass[a4paper,12pt]{article}

% ------------ Load packages -------------
\usepackage{graphicx}
\graphicspath{{fig/}{fig/pdf/}{eps/}{./}}
\usepackage{amsfonts}
\usepackage{amsmath}
\usepackage{array}
\usepackage{float}
\usepackage{multirow}
\usepackage{verbatim}
\usepackage{url}
\usepackage[colorlinks,citecolor=blue]{hyperref}
\usepackage{booktabs}%for nice tabs
\usepackage[small,bf]{caption} %for little font caption
%\usepackage[utf8]{inputenc}
\usepackage[english]{babel}
\usepackage{arydshln}
\usepackage{natbib} % for Bibtek
\bibliographystyle{agu}
\newcommand{\checkthis}[1] {{\textcolor{magenta}{#1}}}


\title{Fully implicit, finite-differences resolution of variably saturated flow in 1D  with Python.}

\author{}

\begin{document}

\maketitle

\section{Soil hydraulic properties}

In unsaturated conditions, water content $\theta$ can be expressed with the closed-form equation provided by \cite{Genuchten1980} :
\begin{equation} \label{eq:genuchten_theta}
    \theta(h) =  \theta_r + (\theta_s - \theta_r) \left( 1+ |\alpha h|^n \right)^{-m}  
\end{equation}
where $\theta_r$, $\theta_s$, $\alpha$ and $n$ are soil characteristics and $m = 1- 1/n$. $h$ [L] is pressure head,counted as negative in unsaturated conditions. For any $h>0$, $\theta(h) = \theta_s$.

The derivative of Eq. \ref{eq:genuchten_theta} with respect to $h$ reads: 
\begin{equation} \label{eq:dtheta_dh}
    \frac{d\theta}{dh}(h) = m n \alpha^n h^{n-1} 
       (\theta_s - \theta_r) \left( 1+ |\alpha h|^n \right)^{-m-1}
\end{equation}
Though apparently different, this expression is equivalent to Eq. 23 in \cite{Genuchten1980}. The derivative was obtained with the condition $h<0$.

Following \cite{Genuchten1980}, the relative permeability $K_r$ is expressed as follows \citep{Genuchten1980}:


\begin{equation} \label{eq:K_r}
    K_r(h) = { { \left( 1-|\alpha h|^{n-1} (1+|\alpha h|^n)^{-m} \right)^2 } \over
		{ (1+ |\alpha h|^{m/2} ) } }
\end{equation}

\section{Model implementation}
We consider the mixed-form equation for variably saturated flow proposed by \cite{Celiaetal1990} and modified by \cite{Clementetal1994}:
\begin{equation} \label{eq:mixed_unsat}
    S_s \frac{\theta}{\eta} \frac{\partial h}{\partial t} + \frac{\partial \theta}{\partial t} = 
    \frac{\partial}{\partial z} \left(-K(\theta) \frac{\partial H}{\partial z} \right) + q
\end{equation}
where $S_s$ $\mathrm{[L^{-1}]}$ is the specific storage, $H$ [L] is the hydraulic head, $\theta$ [-] is the water content, $\eta$ [-] is the porosity, $K$ $\mathrm{[LT^{-1}]}$ is the hydraulic conductivity and $q$ $\mathrm{[T^{-1}]}$ the source term. Contrary to the original Richards' equation, Eq. \ref{eq:mixed_unsat} accounts for the effect of specific storage, which makes it valid for transient saturated flow.

The hydraulic head is defined by $H = h + z$, where $h$ is the pressure head and $z$ the coordinate along the (Oz) vertical upward axis. Eq. \ref{eq:mixed_unsat} yields: 

\begin{equation} \label{eq:mixed_unsat2}
    S_s \frac{\theta}{\eta} \frac{\partial h}{\partial t} + \frac{\partial \theta}{\partial t} =
	 \frac{\partial}{\partial z} \left( K \frac{\partial h}{\partial z} \right) + \frac{\partial K}{\partial z}
\end{equation}

Spatial derivation is obtained with a central differences scheme as follows : 

\begin{align} \label{eq:space_discret}
      \frac{\partial}{\partial z} \left( K \frac{\partial h}{\partial z} \right) + \frac{\partial K}{\partial z} & \approx  \frac{1}{dz} \left\{ \left(\frac{K_i^{n+1,m}+K_{i+1}^{n+1,m}}{2}\right) \left(\frac{h_{i+1}^{n+1,m+1} - h_i^{n+1,m+1}}{dz} \right) \right. \nonumber \\
      &  - \left. \left(\frac{K_i^{n+1,m}+K_{i-1}^{n+1,m}}{2}\right) \left(\frac{h_{i}^{n+1,m+1} - h_{i-1}^{n+1,m+1}}{dz} \right)  \right\} \nonumber  \\
      & + \frac{1}{dz} \left( \left(\frac{K_i^{n+1,m}+K_{i+1}^{n+1,m}}{2}\right) - \left(\frac{K_i^{n+1,m}+K_{i-1}^{n+1,m}}{2}  \right) \right)  
\end{align}
where $n$ denotes the $n$th discrete time level, when the solution is known, $dt$ is the time step, $K_i$ is the value of hydraulic conductivity at the $i$th node and $h_i$ the value of pressure head at the $i$th node. The current and previous Picard iteration are denoted as $m+1$ and $m$, respectively.

A backward finite-difference expression is used for temporal derivative : 
\begin{equation}  \label{eq:dh_temp_deriv}
    S_s \frac{\theta}{\eta} \frac{\partial h}{\partial t} \approx \frac{S_s}{\eta}\theta_i^{n+1,m} \left( { h_i^{n+1,m+1} - h_i^n \over dt} \right)
\end{equation}

After \cite{Celiaetal1990} and \cite{Clementetal1994}, time derivative of water content is approximated as follows:

\begin{equation} \label{eq:theta_temp_deriv}
    \frac{\partial \theta}{\partial t} \approx \left( {\theta_i^{n+1,m} - \theta_i^n \over dt} \right) + C_i^{n+1,m} \left( {h_i^{n+1,m+1}-h_i^{n+1,m} \over dt} \right)
\end{equation}
where $C$ is the derivative of the water content with respect to the pressure head (Eq. \ref{eq:dtheta_dh})

Eqs. \ref{eq:space_discret}, \ref{eq:dh_temp_deriv} and \ref{eq:theta_temp_deriv} can be re-arranged to form a system of linear algebraic equations :

\begin{equation} \label{eq:lin_system}
    m_1 h_{i-1}^{n+1,m+1} + m_2 h_i^{n+1,m+1} + m_3 h_{i+1}^{n+1,m+1} = - b_1 h_i^{n+1,m} - b_2 h_i^n + b_3 + b_4
\end{equation}
with:
\begin{align} \label{eq:coefficients}
    m_1 & = \left(\frac{K_i^{n+1,m}+K_{i-1}^{n+1,m}}{2dz^2}\right)  \\
    m_3 & = \left(\frac{K_i^{n+1,m}+K_{i+1}^{n+1,m}}{2dz^2}\right)  \\
    b_1 & = \frac{C_i^{n+1,m}}{dt} \\
    b_2 & = \frac{S_s\theta_i^{n+1,m}}{\eta dt} \\
    b_3 & = - \left(\frac{K_{i-1}^{n+1,m}-K_{i+1}^{n+1,m}}{2dz}\right) \\
    b_4 & = \left( { \theta_i^{n+1,m} - \theta^n \over dt} \right) \\
    m_2 & = - ( m_1 + m_3 + b_1 + b_2)
\end{align}

The following system should then be solved iteratively : 

\begin{equation} \label{eq:mat_system}
    \mathbf{M}^{n+1,m} \mathbf{h}^{n+1,m+1} = \mathbf{B}^{n+1,m}
\end{equation}
where $\mathbf{M}$ is a sparse matrix containing in its central diagonals the coefficients $m_1$,$m_2$,$m_3$ and vector $\mathbf{B}$ gather the second member of Eq. \ref{eq:lin_system}. 

The resolution of the system is obtained with the Python  \texttt{scipy.sparse.cg}  function.

\section{Boundary conditions}

We consider a soil column subject to infiltration at the top. The boundary condition at the bottom is either a fixed-head or free drainage.

\subsection{Fixed pressure head}

For a fixed pressure head boundary condition at the \textbf{bottom} of the column, the boundary condition is expressed as follows :

\begin{equation}\label{eq:bc_fixed_bot}
    h_1^{n+1,m} = h_{bot}^{n+1}
\end{equation}
where $h_{bot}$ is the value of the imposed pressure head at $t = t_{n+1}$. On the left side of Eq. \ref{eq:lin_system}, we have $m_1 = m_3 = 0$ and $m_2$=1. The right side of Eq. \ref{eq:lin_system} is $h_{bot}^{n+1}$.

Similarly, for a fixed pressure head boundary condition at the \textbf{top} of the column, the boundary condition is expressed as follows :

\begin{equation} \label{eq:bc_fixed_top}
    h_I^{n+1,m} = h_{top}^{n+1}
\end{equation}
where $h_{top}$ is the value of the imposed pressure head at $t = t_{n+1}$. On the left side of Eq. \ref{eq:lin_system}, we have $m_1 = m_3 = 0$ and $m_2$=1. The right side of Eq. \ref{eq:lin_system} is $h_{top}^{n+1}$.


\subsection{Fixed flow}

With a flux $q_z$ $\mathrm{[LT^{-1}]}$ imposed to one of the boundaries of the soil column, Darcy law yields: 
\begin{align} \label{eq:bc_qz}
    q_z = -K \frac{\partial H}{\partial z} &\iff q_z= -K \left( \frac{\partial h}{\partial z} + 1 \right) \nonumber \\
					   &\iff \frac{\partial h}{\partial z} = - \frac{q_z}{K} - 1 
\end{align}

If the flux $q_{top}$ is imposed to the \textbf{bottom} of the column, Eq. \ref{eq:bc_qz} yields:

\begin{equation} \label{eq:bc_qz_bot}
     h_0 = h_1  + \frac{q_{bot}}{K_1} + 1
\end{equation}
where $h_{0}$ is pressure head at a virtual node out of the system. Inserting Eq. \ref{eq:bc_qz_bot} into Eq. \ref{eq:lin_system} yields: 
\begin{equation} \label{eq:bc_qz_lin_system_bot}
    (m_1 + m_2) h_1^{n+1,m+1} + m_3 h_{2}^{n+1,m+1} = - b_1 h_1^{n+1,m} - b_2 h_1^n + b_3 + b_4 + m_3 dz \left( \frac{q_{bot}}{K_1} +1 \right)
\end{equation}

In turn, if the flux $q_{top}$ is imposed to the \textbf{top} of the column, Eq. \ref{eq:bc_qz} yields:

\begin{equation} \label{eq:bc_qz_top}
    h_{I+1}  = h_I  - \frac{q_{top}}{K_I} - 1
\end{equation}

where $I$ is the total number of nodes and $h_{I+1}$ is a virtual node out of the system. Inserting Eq. \ref{eq:bc_qz_top} into Eq. \ref{eq:lin_system} yields: 
\begin{equation} \label{eq:bc_qz_lin_system_top}
    m_1 h_{I-1}^{n+1,m+1} + (m_2 + m_3) h_I^{n+1,m+1} = - b_1 h_I^{n+1,m} - b_2 h_I^n + b_3 + b_4 - m_1 dz \left( \frac{q_{top}}{K_I} +1 \right)
\end{equation}
Note that infiltration into the soil column is simulated with a negative $q_{top}$.


No flow boundary condition is a special case of the fixed flow boundary condition with $q_z = 0$.


\subsection{Free drainage at the bottom of the column}

Free drainage is expressed with a unit vertical hydraulic head gradient: 

\begin{align} \label{eq:bc_drain}
    \frac{\partial H}{\partial z} = 1 &\iff \frac{\partial}{\partial z} \left( z + h \right) =1 \nonumber \\
				      &\iff \frac{\partial h}{\partial z} = 0 \nonumber \\
				      &\iff {\left( h_1 - h_0 \right) \over dz}  = 0 \nonumber \\
				      &\iff h_0 = h_1 
\end{align}

Where $h_0$ is pressure head at a virtual node out of the system. In these conditions, Eq. \ref{eq:lin_system} becomes:

\begin{equation} \label{eq:bc_drain_sys}
    (m_1 + m_2) h_1^{n+1,m+1} + m_3 h_2^{n+1,m+1} = - b_1 h_1^{n+1,m} - b_2 h_1^n + b_3 + b_4
\end{equation}

\section{Flux calculation}
The flux for each node is calculated  when the flow equation is solved at each time. We used the discretized form of the Darcy's Law (eq. \ref{eq:bc_qz}) for all nodes except for the node at the top of the soil profile. At the top node, we solved $q_{top}$ for the mass balance equation and Darcy's Law equation incorporated in eq. \ref{eq:bc_qz_lin_system_top}, as it seems a more stable and mass-conservative solution \citep{Hydrus1d}. Then the node flux is calculated as follows:
\begin{align} \label{eq:flux}
	q_0 &=-K_{1/2}\left(\frac{h_1-h_0}{dz}+1 \right) \nonumber \\
	q_i &=\frac{-K_{i+1/2}\left(\frac{h_{i+1}-h_i}{dz}+1 \right)-K_{i-1/2}\left(\frac{h_{i}-h_{i-1}}{dz}+1 \right)}{2} \\
	q_I &=-K_{I-1/2}\left(\frac{h_{I}-h_{I-1}}{dz}+1 \right)-\frac{dz}{2}\left(\frac{\theta_I^{n+1}-\theta_I^{n}}{dt}+S_p \right) \nonumber
\end{align}



\section{Runoff estimation}
When the rainfall rate exceeds the soil infiltration capacity, a pounding condition occurs at the top of a plane terrain. In time, the pounded water will infiltrate to the soil. Contrary, on sloping terrains the pounding condition is negligable while water is loss as runoff. Following \cite{Herradaetal2014}, the pounding condition is reached when the upper soil surface saturates.
A transient \emph{fixed flow} is prescribed at the upper boundary \citep{Herradaetal2014}:
 \begin{equation} \label{eq:run_pres_inf}
 	In \equiv q(I,n) \equiv q_{rainfall} \equiv q_{top}
\end{equation}
where $In$ is the infiltration rate.
According to \cite{Herradaetal2014}, this condition applies when $\theta(I,n)<\theta_s$, if $\theta(I,n)$ reaches saturation, $\theta(I,n)=\theta_s$.
In our case, as we use the mixed form of the Richard's equation, our boundary conditions are different and the system resolution complex. Following the approach of \cite{Herradaetal2014}, we use the saturation condition from the pressure head $h$ to modify the upper boundary condition. When $h_I^{n+1}<0$, the upper boundary condition is the Eq. \ref{eq:bc_qz_lin_system_top}. In the case where $h_I^{n+1}\geq0$, the fixed flux upper boundary condition switches to a fixed head condition (Eq. \ref{eq:bc_fixed_top}), where $h_{top}^{n+1}=0$. On this scenario, the linear system is resolved again with the new conditions to obtain $h^{n+1}$. The infiltration rate is solved using eq. \ref{eq:flux} for the top node, and surface runoff is calculated as \citep{Herradaetal2014}:
%If we solve the Eq. \ref{eq:bc_qz_lin_system_top} for $q_{top}$ using $h_I^{n+1,m}=h_I^{n+1,m+1}=0$ (a valid assumption after $h_I^{n+1}$ is already known), Eq. \ref{eq:bc_qz_lin_system_top} yields:
%\begin{equation} \label{eq:run_pres_inf}
% 	In=q_{top}=K\left[\frac{\left( m_1h_{I-1}^{n+1}+b_2h_{I-1}^n-b_3-b_4\right) }{\left( m_3dz\right) }-1 \right] 
%\end{equation}
%here $h_{I-1}$ is obtained from the solution of the linear system with the new conditions.
%Finally, surface runoff is calculated as \citep{Herradaetal2014}:
\begin{equation} \label{eq:run_run}
    q_{runoff}=q_{rainfall}-In
\end{equation}

\section{Root water uptake and actual transpiration}
Root water uptake from the soil is used by the plants for transpiration purposes. Total potential transpiration is calculated independently and it is redistributed over the soil profile as follows: 
\begin{equation} \label{eq:rwu_root_dist}
    S_p=b(z)T_p
\end{equation}
where $S_p$ $\mathrm{[T^{-1}]}$ is the potential water uptake rate in a soil unit, $T_p$ $\mathrm{[LT^{-1}]}$ is the total potential transpiration rate and $b(z)$ is the normalized water uptake distribution $\mathrm{[L^{-1}]}$. We considered two types of the $b(z)$ functions:
\begin{itemize}
\item Equally distributed over the root zone:
\end{itemize}
\begin{equation} \label{eq:rwu_bz_1}
    b(z)=\frac{1}{L_R}
\end{equation}
\begin{itemize}
\item Using the function proposed by \cite{Hoffman1983} similar to a trapezoid:
\end{itemize}
\begin{equation} \label{eq:rwu_bz_2}
  b(z)=% 
  \begin{cases}
    \frac{1.667}{L_R} & x\geq L-0.2L_R \\
    \\
    \frac{2.0833}{L_R}(1-\frac{L-z}{L_R}) & L-L_R<z<L-0.2L_R \\
    \\
  	0 & x\leq L-L_R \\
  \end{cases}
\end{equation}
where $L$ is the soil depth and $L_R$ is the root zone depth which is assummed constant. 
The water uptake rate is related to the available water in the soil unit. When not enough water is available, the potential rate cannot be achieved. The actual water uptake is defined by \cite{Feddes1978} as follows:
 \begin{equation} \label{eq:rwu_root_stress}
    S=\alpha(h)S_p
\end{equation}
where $S$ $\mathrm{[T^{-1}]}$ (known in Eq. \ref{eq:mixed_unsat} as $q$) is the actual water uptake and $\alpha(h)$ is a dimensionless stress response function. We used this function related to the water content instead of the pressure head and followed the approach proposed by \cite{FAO1998} modified to soil units:
 \begin{equation} \label{eq:rwu_stress_func}
  \alpha(\theta)=% 
  \begin{cases}
    1 & \theta\geq\theta_{RWC} \\
    \frac{\theta-\theta_{WP}}{\theta_{RAW}} & \theta_{WP}<\theta<\theta_{RWC}\\
    0 & \theta\leq\theta_{WP}
  \end{cases}
\end{equation}
where $\theta_{wp}$ is the water content of the soil unit at the \emph{wilting point} and $\theta_{RWC}$ is the readily extracted water content of the soil unit by the roots, obtained with the equation:
\begin{equation} \label{eq:rwu_root_stress}
    \theta_{RWC}=\rho(\theta_{FC}-\theta_{WP})
\end{equation}
where $\theta_{FC}$ is the water content of the soil unit at the \emph{field capacity}, and $\rho$ is the average fraction of the available extracted water content of the soil unit that can be deplected before moisture stress, and its usually $0.5$ for many crops.
Finally the actual transpiration rate $T_a$ $\mathrm{[LT^{-1}]}$ is calculated with the following equation:
\begin{equation} \label{eq:rwu_act_trans}
    T_a=\sum\limits_{i=1}^{I}\alpha(\theta)b(z)T_p
\end{equation}
Note that $T_a$ in Eq. \ref{eq:rwu_act_trans} do not consider the effect of osmotic stress by the soil salinity nor the compensation by the root adaptability to increase the uptake from other parts when moisture stress occurs. 

\section{Time Step Optimization}
The \emph{time discretization} is associated to the the input time data related to the boundary conditions and the source term, and to the numerical solution. The \emph{time steps} are variable and are calculated automatically by the program in order to reduce the computational time and induce stability to the calculations.
We used the same approach used by \cite{Hydrus1d}. The first time level is the prescribed initial time increment ($\Delta t$) from the input data. Then, $\Delta t$ is automatically modify at each time level as follows \citep{Hydrus1d}:
\begin{equation} \label{eq:time_opt}
  \Delta t^{n+1}=% 
  \begin{cases}
    1.3\Delta t^n & M\leq3 \\
    \\
    \Delta t^n & 3<M\leq7 \\
    \\
  	0.7\Delta t^n & 7<M<M_{max} \\
  	\\
  	\Delta t^n/3 & M=M_{max}
  \end{cases}
\end{equation}
where, $M$ is the number of iteration required to reach covergence at a particular time level and $M_{max}$ is the prescribed maximum iteration number. Eventhough $\Delta t$ is calculated using Eq. \ref{eq:time_opt}, it can be modified in order to coincide with the prescribed time levels in the input data, and to be in the range $[t_{min},t_{max}]$, where $t_{min}$ and $t_{max}$ are prescribed values.
\section{Water balance}
The water balance is calculated for the whole flow region at prescribed time levels.The balance assess the actual water volume and the input flows as infiltration ($In$) and the output flows as percolation ($Pe$), transpiration ($Ta$) and runoff ($Ro$), presenting the total cummulated value of each variable at the time interval.  The water volume is calculated usign:
\begin{equation} \label{eq:wb_vol}
    V_t=\sum\limits_{i=1}^{I-1}\frac{\theta_{i+1}+\theta_i}{2}*dz
\end{equation}
The absolute and relative error in the water balance are calculated as follows:
\begin{align} \label{eq:wb_er}
    \varepsilon_{abs} &=\left(V_{t_1}-V_{t_2}\right)-\left(In-Pe-Ro-Ta\right) \nonumber \\
    \varepsilon_{rel} &=\frac{\varepsilon_{abs}*100}{\left(V_{t_1}-V_{t_2}\right)}
\end{align}

\bibliography{biblio}

\end{document}

